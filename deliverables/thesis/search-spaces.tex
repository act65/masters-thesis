% \chapter{Search spaces}

\newpage

Before going further with our quest for efficient RL. Let's try to understand
some properties of our setting, MDPs.


\section{The value function polytope}

The Value Function Polytope \cite{Dadashi2018} provides some great intuition
about the structure of a MDP and the dynamics and complexity of solvers.
Let's take a look: consider a two state, two action MDP.

\begin{figure}[hb!]
\centering
\includegraphics[width=1\textwidth,height=0.25\textheight]{../../pictures/drawings/2-state-automata.png}
\caption{The simplest possible MDP has two states and two actions. (Any simpler setting is entirely uninteresting. A single state means actions do nothing.
And a single action means all policies are the same.).}
\end{figure}

The space of possible policies is a 2D surface in a 4D space. For each state, we
can pick $\text{action1}$ or $\text{action2}$, with some probability, $p$. For some intuition
about this policy space see \ref{high-D-policies}.

\begin{align}
\pi &=
\begin{bmatrix}
  p(a=a_1|s=s_1) & p(a=a_2|s=s_1) \\
  p(a=a_1|s=s_2) & p(a=a_2|s=s_2)\\
\end{bmatrix} \\
&=
\begin{bmatrix}
p(a=a_1|s=s_1) & 1-p(a=a_1|s=s_1) \\
p(a=a_1|s=s_2) & 1-p(a=a_1|s=s_2)\\
\end{bmatrix}
\end{align}

Since the policies are a 2D space, we can visualise them. This square of all possible policies is not particularly interesting.

Rather, we can evaluate (calculate the expected return) each each policy (using the \eqref{eq:value-functional}).
Since there are two states, the evaluation returns a 2D vector of values, one value for each state.
Therefore, we can visualise the value of each policy.
\begin{figure}[!hb]
\centering
\includegraphics[width=1\textwidth,height=0.5\textheight]{../../pictures/figures/value-polytope.png}
\caption{For every policy, we can plot a dot where that value of that policy lies in 'value space'.
The red dots are deterministic policies.}
\end{figure}

Dadashi et al. \cite{Dadashi2018} explored a few properties of the polytope.
Specifically they focused on its geometry and dynamics. In
\ref{polytope-extras} you can find further exploration of other properties of
the value polytope, such as; the density of policies and the effect of the discount rate.

\subsubsection{Geometry of the polytope}

Dadashi et al. remark, the polytope gives a clear illustration of the following classic results regarding MDPs \cite{Bertsekas1996}.

\begin{enumerate}
\tightlist
  \item (Dominance of $V^*$) The optimal value function $V^*$ is the unique dominating vertex of $V$;
  \item (Monotonicity) The edges of V are oriented with the positive orthant;
  \item (Continuity) The space V is connected.
\end{enumerate}

\textbf{1)} {\color{red} that isnt correct!?}

\textbf{2)} If $V(s_2)$ increases then $V(s_1)$ must either increase or stay the same.
This can be seen in the last equation below;

\begin{align*}
V(s_1) &= \mathop{\mathbb E}_{a \sim\pi(\cdot|s_1)} r(s, a) + \gamma \mathop{\mathbb E}_{s'\sim \sum_a P(\cdot|s, a)\pi(\cdot|s)} V(s')\\
&= \sum_a \pi(a|s_1)r(s, a) + \gamma \sum_{s'}\sum_a P(s'|s_1, a)\pi(a|s) V(s') \\
&= \sum_a \pi(a|s_1)r(s, a) + \gamma \sum_a P(s_1|s_1, a)\pi(a|s) V(s_1) + \gamma\sum_a P(s_2|s_1, a)\pi(a|s) V(s_2)
\end{align*}

If $\sum_a P(s_2|s_1, a)\pi(a|s) = 0$ then $V(s_1)$ stays the same, yielding a constant vertical line on the polytope.
If $\sum_a P(s_2|s_1, a)\pi(a|s) > 0$ then $V(s_1)$ increases with $V(s_2)$, yielding a positive orthant.
$\sum_a P(s_2|s_1, a)\pi(a|s) < 0$ is not possible.

\textbf{3)} The policy space is connected, and the value function is continuous.
Therefore the value space, the polytope, is connected. {\color{red}should give more details?!}

% Further more, Dadashi et al show that ... Line theorem. s-deterministic policies.

\subsubsection{Dynamics on the polytope}

Furthermore, Dadashi et al. \cite{Dadashi2018} were interested in three aspects of different algorithms’ learning dynamics:

\begin{itemize}
\tightlist
  \item the path taken through the value polytope,
  \item the speed at which they traverse the polytope,
  \item and any accumulation points that occur along this path.
\end{itemize}

% Why do they care?
% How quickly do these learners traverse the polytope? Do algorithms take the shortest path? Where are the accumulation points?

They consider value iteration, policy iteration, policy gradients, entropy regularized policy gradients,
natural policy gradients and the cross entropy method.

Their results are intriguing. They show that different RL algorithms traverse the polytope in vastly different ways.
Some are not even constrained to the polytope. This left me wondering;

{\color{red}I feel like this needs more???}

\begin{displayquote}
  \textit{How does a search algorithm interact with its search space to yield efficient search?}
\end{displayquote}

\section{Search spaces for MDPs}\label{search-spaces-mdps}

We want to efficiently find the optimal policy for a given MDP. But where and how should we
search for this policy? We could search within;

\begin{itemize}
\tightlist
  \item the set of potentially optimal policies, the $|A|^{|S|}$ discrete policies,
  \item the set of all possible policies $\pi \in \mathbb R^{|S| \times |A|}: \forall s \int_a \pi(a|s) = 1$
  \item the set of possible state-action value functions, $\mathbb R^{|S|\times|S|}$,
  which we could then use to construct the optimal policy,
  \item Or maybe some other space.
\end{itemize}

\begin{displayquote}
  \textit{Which space is best? Which space allows us to find the optimal policy in the 'cheapest' manner?}
\end{displayquote}

Naively, we think smaller search spaces are better. We would rather
search for our keys in a few rooms, rather than many. But added
structure (for example, an ordering) can be exploited to yield faster
search, even when there are infinitely more states to search. For example,
we might be able to order the rooms based on how recently we visited them.
This should help us retrace our steps and find our keys, rather than arbitrarily
picking rooms to search.

\subsection{Policy search}

We can search through policies. In my opinion, this feels like the most 'natural' type of search for RL.
As, after all, we are searching for the optimal \underline{policy}.

Searching through the space of policies supports a couple of modes of travel:
policy iteration and policy gradients

\subsubsection{Policy iteration}

In policy iteration, we search for the optimal policy by evaluating our current
policy and then acting greedily. In our tabular setting, policy iteraction can be written as:

\begin{algorithm}
\caption{Policy iteration}
\begin{algorithmic}[1]

\Procedure{PI}{$P, r, \gamma$}
    \State $\pi_t \sim \mathcal U(\Pi)$
    \While{not converged}
      \State $V_t = (I-\gamma P_{\pi_t})^{-1} r_{\pi_t}$ \Comment{Evaluate policy}
      \State $Q_t =  r + \gamma P\cdot_{s'} V_t$ \Comment{Bellman operator}
      \State $\pi = \text{greedy}(Q_t) $ \Comment{Greedy update}
    \EndWhile
    \State \algorithmicreturn{ $\pi$}
\EndProcedure

\end{algorithmic}
\end{algorithm}

The greedy operator picks the actions that give the highest state-action return,
and sets their probability to be $1$.
$\text{greedy}(Q) = \text{onehot}(\text{argmax}_a Q[s, a], |A|)$.

This iteration converges because the state-action values capture counterfactuals.
\textit{What would the return be if I took an action, $a$, not necessesarily chosen by
the current policy, but then followed the current policy afterward. $Q^{\pi}(s, a)$}
If there exists an action that achieves higher return than the current policies choice,
then (because of the greedy step) PI will update the policy so it chooses that action.

\footnotemark[7]

\footnotetext[7]{Actually has connections to the simplex method. ref}

{\color{red}ref for PI.}

\begin{figure}[h!]
\centering
\includegraphics[width=0.8\textwidth,height=0.4\textheight]{../../pictures/figures/pi-polytope.png}
\caption{PI jumps between the deterministic policies (the vertices of our polytope).
This is because of the greedy step taken.}
\end{figure}

\newpage

\subsubsection{Policy gradients}

This search is closely related to the deep learning / end-to-end paradigm.
\textit{Simply write down what you want (the loss function),
estimate its derivative and apply gradient descent.}

In our case, the loss function is the value. And, we can estimate the derivative
by differentiating the value functional \ref{eq:value-functional} with respect to the policy.

To ensure the optimisation problem is constrained properly, we pick $\theta$ as our parameters and
construct the policy using $\pi = \sigma(\theta)$, where $\sigma$ is the softmax function.

% However, has problems with sparse rewards / exploration (not considered in this setting).
% (but dont they all have problems with sparse rewards!?)

\begin{algorithm}
\caption{Policy gradients}
\begin{algorithmic}[1]

\Procedure{PG}{$P, r, \gamma, \eta$}
  \State $t=0, \quad\quad \theta_t = \log(\pi) \quad\quad \pi \sim \mathcal U(\Pi)$ \Comment{Init}
  \While{not converged}
    \State $\theta_{t+1} = \pi_t + \eta \nabla_{\theta} V(\sigma(\theta_t))$ \Comment{Gradient update}
    \State t += 1
  \EndWhile
  \State \algorithmicreturn{ $\sigma(\theta_t)$}
\EndProcedure

\end{algorithmic}
\end{algorithm}

Note that to mitigate stability issues, we also added weak regularisation
attempting to maximise the entropy of the policy. This forces the policies away
from the edges of the polytope, where the gradients are not defined ($\log(0) = \text{NaN}$).

\begin{figure}[h!]
\centering
\includegraphics[width=0.8\textwidth,height=0.4\textheight]{../../pictures/figures/pg-polytope.png}
\caption{An example of ... . Note that the majority of }
\end{figure}

% Want to include upper / lower bounds!?
{\color{red}??? Converges at a rate of $\frac{1}{t}$. As can be seen by ...}
\cite{Agarwal2019a}.

\newpage

\subsection{Value search}

Alternatively, we can search through possible values. But how can we ensure that our search will
converge to a value that corresponds to a realisable policy? We can use Bellman's
optimality operator to guide the search.

(For similar reasons to why policy iteration converges) The greedy step using the
state-action values will find actions with higher value.


% (need to explain? why does stationarity mean optimality...?!)
Intution about why it converges!? Contraction. Banach fixed-point theorem.

% \begin{align}
% T(V) &= \mathop{\text{max}}_a \big[r + \gamma PV\big] \\
% \end{align}

\begin{algorithm}
\caption{Value iteration}
\begin{algorithmic}[1]

\Procedure{VI}{$P, r, \gamma, \eta$}
  \State t = 0
  \State $V_t = V(\pi) ; \;\; \pi \sim \mathcal U(\Pi)$ \Comment{Init}
  \While{not converged}
    % \State $V = r + \gamma PV_t$ \Comment{Evaluate}
    \State $Q = r + \gamma PV_t$ \Comment{Bellman operator}
    \State $\hat V = \text{max}_a Q(s, a)$
    \State $V_{t+1} = V_t + \eta (\hat V - V_t)$ \Comment{Average}
    \State t += 1
  \EndWhile
  \State $\pi = \mathop{\text{argmax}}_{\pi} r_{\pi} + \gamma P_{\pi}V_t$
  \State \algorithmicreturn{ $\pi$}
\EndProcedure

\end{algorithmic}
\end{algorithm}

\begin{figure}[h!]
\centering
\includegraphics[width=0.8\textwidth,height=0.4\textheight]{../../pictures/figures/vi-polytope.png}
\caption{Observe that the value iterations are not constrained to map to a policy
(they can go out side of the polytope), but they do converge to a realisable policy,
the optimal one.}
\end{figure}

{\color{red}important. Also, why does it go outside?!?}
% Want to include upper / lower bounds!? On complexity. Sample / computational!?

\newpage

\begin{center}\rule{0.5\linewidth}{\linethickness}\end{center}

So there are different classes of search space: each imbued with special
structure from the Bellman equation or expected return. Each with different types of search they
support.

\begin{displayquote}
\textit{Which spaces support efficient search for the optimal policy? Can we characterise the properties of each space?}
\end{displayquote}

See appendix \ref{ss-extras} for an experimental exploration of the iteration complexity and dynamics of these different alforithms.
{\color{red}want to include in main section. but needs more work. not really sure what i am trying to say...}
