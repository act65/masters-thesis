\chapter{MDPs}

\subsection{MDP examples}
Examples of MDPs in the wild.

\hypertarget{bus-engine-replacement}{%
\paragraph{Bus engine replacement}\label{bus-engine-replacement}}

\begin{itemize}
\tightlist
\item
  \textbf{States:} Accumulated mileage of each bus (since their last
  replacement).
\item
  \textbf{Actions:} Replace bus \(i\), Y / N. If Y, what year model to
  replace with?
\item
  \textbf{Transition fn:} How the mileage of each bus changes between
  fitness checks.
\item
  \textbf{Reward fn:} Age dependent recurring cost - for repairs - and
  replacement cost.
\end{itemize}

Note: The transition function could be diagonal, or not. If diagonal
then buses are not able to effect the milage of other busses, possibly
by taking another's shift. In this case, the problem is reduced to a
contextual bandit problem (?).

\cite{Putterman2015}

\hypertarget{the-aloha-protocol}{%
\paragraph{The ALOHA protocol}\label{the-aloha-protocol}}

\begin{itemize}
\tightlist
\item
  \textbf{States:} For each terminal, is last attempt a collision.
\item
  \textbf{Actions:} The probability of each terminal attempting to send
  a new packet (if there has been a collision).
\item
  \textbf{Transition fn:} Combines terminal packets into either a
  successful transmission, or a collision.
\item
  \textbf{Reward fn:} If a packet is sent, great. If not, bad.
\end{itemize}

\cite{Putterman2015} pg 8

\hypertarget{mate-desertion-in-coopers-hawks}{%
\paragraph{Mate desertion in Cooper's
Hawks}\label{mate-desertion-in-coopers-hawks}}

\begin{itemize}
\tightlist
\item
  \textbf{States:} The product of the brood's health and mother's health
  (\([2:7] \times [2:7]\)).
\item
  \textbf{Actions:} Stay, hunt, desert.
\item
  \textbf{Transition fn:} The four developmental stages, early nestling,
  late netling, early feldgling, late feldgling. From one developmental
  stage to the next, the energy levels of the mother and brood are
  determined by the initial energy reserves, the actions taken and the
  availability of food. (this was estimated from data data gatherd by
  \ldots{})
\item
  \textbf{Reward fn:}
\end{itemize}

\cite{Putterman2015} pg 10

\hypertarget{but-whos-counting}{%
\paragraph{But who's counting?}\label{but-whos-counting}}

\begin{itemize}
\tightlist
\item
  \textbf{States:} A random number, and the value of each of five
  possible locations. Possibly none value.
\item
  \textbf{Actions:} Choose which location to add the latest random
  number.
\item
  \textbf{Transition fn:} Deterministically updates the storage location
  given the action and observed random number.
\item
  \textbf{Reward fn:} The total magnitude of the stored number, only
  given after the storage is full..
\end{itemize}

\cite{Putterman2015} pg 13

\hypertarget{diagnosing-catnip-immunity}{%
\paragraph{Diagnosing catnip
immunity}\label{diagnosing-catnip-immunity}}

\begin{itemize}
\tightlist
\item
  \textbf{States:} The truth values for immuity to an of the 4 drugs (
  Catnip / Valerian / Silvervine / Honeysuckle )
\item
  \textbf{Actions:} Choose which drug to test.
\item
  \textbf{Transition fn:} Updates the truth values with some probability
  of returning a true postive or false negative.
\item
  \textbf{Reward fn:} Minimize the cost to find a working drug.
  Catnip=\$8.96, Valerian=\$7.00, Silverine=\$17.77, Honeysuckle=\$7.99.
\end{itemize}

\begin{quote}
Bol et al 2017, as noted, provides responses for 4 drugs
(catnip/Valerian/silvervine/honeysuckle) in a large sample of cats;
responses turn out to be heavily intercorrelated, permitting the ability
to better predict responses to the catnip alternatives based on a known
response to one of the others. This becomes useful if we treat it as a
drug selection problem where we would like to find at least one working
drug for a cat while saving money, and adapting our next test based on
failed previous tests.
\end{quote}

\begin{quote}
If they were not intercorrelated, one would simply minimize expected
loss in a greedy fashion, starting with catnip etc; but as they are
intercorrelated, now a drug has both direct value (if the cat responds)
and value of information (its failure gives evidence about what other
drugs that cat might respond to), which means the greedy policy may no
longer be the optimal policy.
\end{quote}

\href{https://www.gwern.net/Catnip\#optimal-catnip-alternative-selection-solving-the-mdp}{gwern
on Catnip}

This one is interesting. The four actions effect only the four state
element-wise. But our knowledge that certain immunities are correlated
make it possible to intelligently guess which tests should be performed.

\hypertarget{ad-targeting}{%
\paragraph{Ad targeting}\label{ad-targeting}}

\hypertarget{youtube-recommendation}{%
\paragraph{Youtube recommendation}\label{youtube-recommendation}}

\hypertarget{salamon-harvesting}{%
\paragraph{Salamon harvesting}\label{salamon-harvesting}}

\begin{itemize}
\tightlist
\item
  \textbf{States:} The size of the salamon population.
\item
  \textbf{Actions:} The size of the salamon population to be left to
  spawn.
\item
  \textbf{Transition fn:} Given the number left to spawn, it returns the
  size of the salamon population in the next season.
\item
  \textbf{Reward fn:} The size of salamon population harvested.
\end{itemize}

YOu might call this MDP a one dimensional MDP, as there is only a single
dimension that is acted up, is transitioned, is rewarded\ldots{} More
salamon

\href{http://www.it.uu.se/edu/course/homepage/aism/st11/MDPApplications1.pdf}{Real
applications of MDPs}

\hypertarget{fire-engine-allocation}{%
\paragraph{Fire engine allocation}\label{fire-engine-allocation}}

\begin{itemize}
\tightlist
\item
  \textbf{States:} The magnitude of a fire. The type of alarm. And the
  total number of first and second fire engines already deployed.
\item
  \textbf{Actions:} Whether to send more fire engines.
\item
  \textbf{Transition fn:} Given the number of fire engines fighting the
  fire, and the fires type / magnitude, the building may be destroyed or
  saved. Fire may start at anytime, in a random location throughout the
  city.
\item
  \textbf{Reward fn:} Damage incurred by the fires.
\end{itemize}

\href{http://www.it.uu.se/edu/course/homepage/aism/st11/MDPApplications1.pdf}{Real
applications of MDPs}

Other possible MDPs?

\begin{itemize}
\tightlist
\item
  An animal stockpiling food?!
\item
  Robotics / movement
\end{itemize}

\begin{center}\rule{0.5\linewidth}{\linethickness}\end{center}

More Refs

\begin{itemize}
\tightlist
\item
  http://www.it.uu.se/edu/course/homepage/aism/st11/MDPApplications1.pdf
\item
  https://www.worldscientific.com/worldscibooks/10.1142/p809
\end{itemize}




\subsection{Derivation of derivative}

\begin{align}
V(\pi) &= (I − \gamma P_{\pi})^{−1}r_{\pi} \\
&= (I − \gamma P\cdot \pi)^{−1}r\cdot \pi \\
\frac{\partial V}{\partial \pi} &= \frac{\partial}{\partial \pi}((I-\gamma P_{\pi})^{-1} r_{\pi}) \\
&= (I-\gamma \pi P)^{-1} \frac{\partial \pi r}{\partial \pi}+   \frac{\partial (I-\gamma \pi P)^{-1}}{\partial \pi}\pi r\tag{product rule} \\
&= (I-\gamma \pi P)^{-1} r + -(I-\gamma \pi P)^{-2} \cdot -\gamma P\cdot \pi r\\
&= \frac{r}{I-\gamma \pi P} + \frac{ \gamma P\cdot \pi r}{(I-\gamma \pi P)^2}\\
&= \frac{r(I-\gamma \pi P) + \gamma P \pi r}{(I-\gamma \pi P)^2} \\
& = \frac{r}{(I-\gamma P \pi)^2} \\
\end{align}

TODO. Annotate. Add numbers and explain.
