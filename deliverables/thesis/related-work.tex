\chapter{Related work}

How do the topics considered in this thesis relate to the work done in the wider scientific community? Which work uses similar tools, which works build on the same foundations, which works have the same goals?

\hypertarget{related-work}{%
\section{Related work}\label{related-work}}

\hypertarget{mdps}{%
\paragraph{MDPs}\label{mdps}}

Dynamic programming, linear programming, \ldots{}?

\[
Q^{\pi}(s_0, a_0) = r(s_0, a_0)
+ \gamma \mathop{\text{max}}_{a_1} \mathop{\mathbb E}_{s_1\sim p(\cdot | s_0, a_0)} \Bigg[ r(s_1, a_1)
+ \gamma \mathop{\text{max}}_{a_2} \mathop{\mathbb E}_{s_2\sim p(\cdot | s_1, a_1)} \bigg[r(s_2, a_2)
+ \gamma \mathop{\text{max}}_{a_3} \mathop{\mathbb E}_{s_3\sim p(\cdot | s_2, a_2)} \Big[
\dots \Big] \bigg] \Bigg]
\]

\hypertarget{hrl}{%
\subparagraph{HRL}\label{hrl}}

Temoral abstractions of actions.(how does this related to a
decomposition of rewards) Ok, so we wany a multiscale representation?
Understanding how actions combine (this is necessary knowledge for HRL?)

Reasons to do HRL??? (want to verify these claims - and have refs for
them)

\begin{itemize}
\item
  credit assignment over long time periods (learning faster in one env)
\item
  exploration
\item
  transfer
\item
  To learn action abstractions they must capture info about the model.
  How much harder is it to learn action abstractions in model-free vs
  model-based settings?
\item
  Reward as a function of a subspace of the state space. (this is
  important for learning abstract representations and actions!?)
\item
  What do cts linear heirarchical actions look like!? and their loss
  surface!?
\item
  \href{https://arxiv.org/abs/1612.02757}{HLMDPs}
\item
  \href{https://arxiv.org/abs/1812.00025}{Modulated policy heirarchies}
\item
  \href{https://arxiv.org/abs/1810.10096}{Model free representations for
  HRL}
\item
  \href{https://blog.aqnichol.com/2019/04/03/prierarchy-implicit-hierarchies/}{Prierarchy:
  Implicit Hierarchies}
\item
  Options
\item
  \href{https://openreview.net/forum?id=H1emus0qF7}{Near optimal
  representation learning for heirarchical RL}
\end{itemize}

Relation to pretraining / conditioning?

\begin{center}\rule{0.5\linewidth}{\linethickness}\end{center}

Why does Heirarchy (sometimes) work so well in reinforcement learning?

The authors claim that the benefits of HRL can be explained by better
exploration. However, I would interpret their results as saying; ``for
2D environments with walls, larger steps / actions result in greater
explration''. But what if the walls were replaced by cliffs? I imagine
this algorithm would do a lot worse!?

They also seem to misunderstand the main problem with HRL, discovery.
Once you have discovered a nice set of abstracted actions / a
representation, then yeah, you get faster reward propagation, better
exploration, \ldots{} etc.

\hypertarget{dynamic-programming}{%
\paragraph{Dynamic programming}\label{dynamic-programming}}

What is it? Memoized search. Why should we care?

\hypertarget{model-based-rl}{%
\subsection{Model-based RL}\label{model-based-rl}}

Pros and cons.

Model-based learning can be bad\ldots{} There may be many irrelevant
details in the environment that do not need to be modelled. A model-free
learning naturally ignores these things.

The importance of having an accurate model!

For example, let \(S\in R^n\) and \(A\in [0, 1]^n\). Take a transition
function that describes how a state-action pair generates a distribution
over next states \(\tau: S \times A \to \mathcal D(S)\). The reward
might be invariant to many of the dimensions.
\(r: X \times A -> \mathbb R\), where \(X \subset S\).

Thus, a model mased learner can have arbitrarily more to learn, by
attempting to learn the transition function. But a model-free learner
only focuses on \ldots{}

This leads us to ask, how can we build a representation for model-based
learning that matches the invariances in the reward function. (does it
follow that the invariances in reward fn are the invariances in the
value fn. i dont think so!?)

Take \(S \in R^d\) and let \(\hat S = S \times N, N \in R^k\). Where
\(N\) the is sampled noise. How much harder is it to learn
\(f: S \to S\) versus \(\hat f: \hat S \to \hat S\)?

https://arxiv.org/pdf/1903.00374v3.pdf https://arxiv.org/abs/1907.02057

\hypertarget{representation-learning-and-abstraction}{%
\subsection{Representation learning and
abstraction}\label{representation-learning-and-abstraction}}

The goal is to find a representation that decomposes knowledge into its
parts.

Another way to frame this is: trying to find the basis with the right
properties.

\begin{itemize}
\tightlist
\item
  sparsity,
\item
  independence,
\item
  multi scale,
\item
  locality/connectedness
\item
  ???
\end{itemize}


Types of abstraction for RL. Abstraction for efficient;

\begin{itemize}
\tightlist
\item
  exploration, [Learning latent state representation for speeding up exploration](https://arxiv.org/abs/1905.12621)
\item
  optimal control,
\item
  ???,
\end{itemize}


\subsection{Heirarchical reinforcement learning}
