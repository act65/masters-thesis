\newpage
\section{Action abstractions}


\epigraph{
A green leaf, is too far, out of reach,
What you want, is in front, take the steps.

You move your first leg up You move your second leg left ... You move your eleventh leg up You move your twelfth leg right

Many legs burden the act,
Unless coordinated in abstract.

The abstract words in this new language,
Must be as few as we can manage.

"left", "right", "forward", "backward"

This time, the act is rather brief,
"forward", to reach the tasty leaf.
}{\textit{The writer and poet: Alex Telfar}}


Action abstraction is possibly the least explored type of abstraction.
There exists a lot of work seeking state abstraction (refs), and heirarchical (temporal) abstraction (refs)
but, this has only recently been some work ...

\subsection{Interfaces}

The intution is.
People can generalise / adapt between different action spaces.

\begin{itemize}
\tightlist
\item
  Might be teleported to a new environment? (new state space, same
  action space)
\item
  Might have to drive a new vehicle (same state space, new action space)
\end{itemize}


Utility?
- Adapt to broken limbs
- To new ?

% ref sergey levine meta adaptation limb paper!?



% \begin{figure}
% \centering
% \includegraphics[width=0.5\textwidth,height=0.5\textheight]{../../pictures/figures/discrete-interface.png}
% \caption{The optimisation dynamics of value iteration versus parameterised value iteration.}
% \end{figure}


\subsection{Related work}

\cite{Nagabandi2019}


% More generally. Connection between sample / computational complexity via
% SGD + regularisation -> Langevin dynamics -> rejection sampling???

% \subsubsection{Langevin dynamics}
%
% Clip to nearest symmetric abstraction.
%
% \begin{align}
% \mathop{\text{argmin}}_{\theta} D(\bar \theta, \theta)
% \end{align}
%
% Where $\bar \theta$ is the nearest symmetric set of parameters.
%
% \begin{align}
% \theta = \theta_t - \eta \nabla_\theta \mathcal S(\theta) + w
% \end{align}
%
% Langevin dynamics. Where $w$ is sampled from some noise distribution.
% Therefore, we get a distribution over $\theta$.
% Only do a few iterations, so we sample $\theta$ that are near by!?!?
% The $\nabla_\theta \mathcal S(\theta)$ pulls us towards symmetric parameters.
%
% Need a differentiable measure of symmetry.!?
% If I actually find this, we can just use it as a regulariser..!?!?!?





% \subsection{Pairing abstractions with similarity measures}
%
% % we have a similarity measure and a way to allow the learner to use it. now ...
%
% Want to remove unimportant structure, while keeping the structure that allows us
% to exploit the bellman equation (and thus more efficient search).
%
% Different abstraction spaces support different similarity measures.
% State similarity based on value fails in rather simple settings.
% But works with state-action abstractions!?
%
% Weaker notions of similarity.
% \begin{itemize}
% \tightlist
%   \item Preserving the ordering of policies. Rather than their absolute value.
%   \item
%   \item Preserving the neighbourhoods of policies and their values.
%   \item ?
% \end{itemize}
%








$S_4$

 \begin{align*}
  \begin{bmatrix}
    0 & 1 & 2 & 3 \\
    3 & 0 & 1 & 2 \\
    2 & 3 & 0 & 1 \\
    1 & 2 & 3 & 0 \\
  \end{bmatrix}
  \begin{bmatrix}
    a & b & c & d \\
    d & a & b & c \\
    c & d & a & b \\
    b & c & d & a \\
  \end{bmatrix}
 \end{align*}

$S_2 \times S_2$

 \begin{align*}
  \begin{bmatrix}
    (0, 0) & (0, 1) & (1, 0) & (1, 1) \\
    (0, 1) & (0, 0) & (1, 1) & (1, 0) \\
    (1, 0) & (1, 1) & (0, 0) & (0, 1) \\
    (1, 1) & (1, 0) & (0, 1) & (0, 0) \\
  \end{bmatrix}
 \begin{bmatrix}
   a & b & c & d \\
   b & a & d & c \\
   c & d & a & b \\
   b & c & b & a \\
 \end{bmatrix}
 \end{align*}

 % http://mathworld.wolfram.com/FiniteGroup.html

How many observations are sufficient to discriminate between $S_4$ and $S_2\times S_2$?
What about symmetries of different order?
