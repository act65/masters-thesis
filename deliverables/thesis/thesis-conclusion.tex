\chapter{Discussion and conclusion}\label{C:con}

If all the economists in the world were laid end-to-end they wouldn't
reach a conclusion, and neither shall I.


\section{Discussion}

The LMDPs attempted to preserve, ... . But failed to preserve X, and thus cannot guarantee performance in general.

Symmetry attempts to preserve, ...?

Need better tools for this!?


\section{Future work}

There is a large amount of future work to be done if we want to understand abstractions for efficient RL.

In \ref{exploit-abstraction-rl} we described \textit{state abstraction}, \textit{action abstraction}, and \textit{state-action abstraction}, but what is the advantage, if any, of \textit{state and action abstraction} versus \textit{state-action abstraction}? Similarity, \ref{exploit-abstraction-rl} we described \textit{goal-like} and \textit{option-like} temporal abstraction. But, is one strictly better that the other. If not, then in which cases does \textit{goal-like} temporal abstraction perform better?

Is the family of similarity notions presented in \ref{similar-classes} 'complete'? Does it describe all abstractions of interest?
Relatedly, are cumulants strictly superior to trajectory type similarity measures $\chi(x, x') = \int_\pi \mathbb E \sum D(c(x, \pi), c(x', \pi)))$?

What is the trade off of approximating $\int_{\pi \in \Pi}f(\pi)$ with $\int_{\pi \in B \subset \Pi}f(\pi)$?
How large does $X$ need to be to get a reliable estimate? Can we pick $X$ in intelligent, or random ways.
The topology of abstractions (partial ordering). Seems important!? Can use somehow?
hat is necessary for the preservation of Bellman equations ability to guide search? Can we weaken the requirements to only preserving the ordering of the value of optimal actions?

% The comparison of topologies constructed using varios k length options.
% With increaseing k, the topology of the abstraction is always coarser??
% Are these properties for evaluation necessary and / or sufficient for good performance of an abstraction?


Questions / future work
\begin{itemize}
	\item If we know the order of the group, then how hard is it to discover a symmetry?
\end{itemize}

Future work.
Metropolis-hastings. To allow this to scale to higher dimensions.


Future work.
- Generalise to more symmetries.
- How do the number of actions scale with d and / or the (type of) symmetry group?
- learn representations that are ordered. So we can reduce the combinatorial space of possible symmetries.

Future work.
Use this measure with SGD as a regualriser.

\section{Conclusion}
